\documentclass{article}
\usepackage[utf8]{inputenc}

\usepackage[T2A]{fontenc}
\usepackage[utf8]{inputenc}
\usepackage[russian]{babel}

\usepackage{multienum}
\usepackage{geometry}
\usepackage{hyperref}

\geometry{
 left=1cm,right=1cm,
 top=2cm,bottom=2cm
}

\title{История}
\author{Лисид Лаконский}
\date{May 2023}

\newtheorem{definition}{Определение}

\begin{document}
\raggedright

\maketitle
\tableofcontents
\pagebreak

\section{Практическое занятие по истории №16, «Социально-экономическое развитие СССР в 1953–1964 годы»}

\subsection{Экономика страны в начале 50-х годов}

В 50-е годы СССР входил в число стран с наиболее высокими темпами экономического роста наряду с ФРГ, Японией, Францией и некоторыми другими странами, значительно опережая темпы экономического роста в США и Великобритании, и многих других стран мира.

Смотри таблицу №1: \href{https://istmat.org/node/57531}{статья}

\hfill

Рост ВВП в СССР в целом за весь период 50-х годов многократно превосходил рост в таких странах, как США и Великобритания, значительно опережал экономический рост во Франции, был выше, чем в ФРГ и лишь незначительно уступал экономическому росту в Японии.

Исключительно высокие темпы экономического роста в СССР, в данный период, видны и при сравнении динамики роста важнейшей отрасли экономики — промышленности

Смотри таблицу №2: \href{https://istmat.org/node/57531}{статья}

\hfill

Принципиально новым обстоятельством для этого периода в развитии советской экономики было то, что, в отличие от предыдущего периода, интенсивные факторы стали основными в развитии экономики. Так, при росте ВВП более чем на 100\%, численность занятых выросла за 50-е годы лишь на 22\%.Таким образом, за счет роста производительности труда обеспечивалось более 80\% прироста ВВП, в то время как до войны менее половины.

\hfill

50-е годы характеризуются исключительно быстрыми структурными сдвигами в экономике. Быстро растет урбанизация страны, развитие новых отраслей экономики /производство электронно-вычислительной техники, многих отраслей приборостроения, химической промышленности, коренная техническая реконструкция железнодорожного транспорта, развитие авиационного транспорта, возникновение и развитие производства редких металлов, системы научных учреждений в разных областях науки и техники/.

\hfill

СССР в этот период осуществлял очень значительное финансово-кредитное и научно-техническое содействие своим союзникам в восточной Европе и в Китае /до 1960 г./, начал разворачивать значительное содействие в экономическом развитии ряду развивающихся стран, наиболее заметными проявлениями которого были сооружаемые на высоком техническом уровне и быстро такие объекты, как Бхилайский металлургический завод в Индии и Асуанская плотина в Египте.

\hfill

Продолжалось интенсивное наращивание вложений в развитие образования, здравоохранения и науку, которое приняло огромные размеры уже в довоенный период. Лучше всего о размерах этого наращивания говорят увеличение расходов на эти цели из государственного бюджета, которые ввиду незначительного роста цен практически совпадали с реальными вложениями в эти отрасли. Так, расходы бюджета на просвещение выросли с 5,7 млрд. руб. в 1950 г. до 10,3 млрд. руб. в 1960 г., на нужды здравоохранения и физической культуры с 2,1 млрд. руб. в 1950 г. до 4,8 млрд. руб., т.е. по обеим отраслям в 2-2,5 раза.По доле расходов на образование, здравоохранение и науку в ВВП в этот период СССР, как известно, занимал одно из самых высоких мест в мире.

\hfill

Серьезнейшим экономическим достижением 50-х годов явилась невиданная для СССР и редко встречавшаяся в 20 веке вообще финансовая стабилизация, выражавшаяся в профицитном бюджете и минимальном росте розничных и оптовых цен, и даже их сокращении в начале 50-х годов.

\pagebreak
\subsection{В поисках новой стратегии развития: «генеральная линия» 1954 года и проекты Берии, Маленкова и Хрущева}

\href{https://ru.wikipedia.org/wiki/%D0%A0%D0%B5%D1%84%D0%BE%D1%80%D0%BC%D1%8B_%D0%91%D0%B5%D1%80%D0%B8%D0%B8}{Реформы Берии}

\hfill

\href{https://ru.wikipedia.org/wiki/%D0%91%D0%B5%D1%80%D0%B8%D1%8F,_%D0%9B%D0%B0%D0%B2%D1%80%D0%B5%D0%BD%D1%82%D0%B8%D0%B9_%D0%9F%D0%B0%D0%B2%D0%BB%D0%BE%D0%B2%D0%B8%D1%87#%D0%A1%D0%BC%D0%B5%D1%80%D1%82%D1%8C_%D0%A1%D1%82%D0%B0%D0%BB%D0%B8%D0%BD%D0%B0._%D0%91%D0%BE%D1%80%D1%8C%D0%B1%D0%B0_%D0%B7%D0%B0_%D0%B2%D0%BB%D0%B0%D1%81%D1%82%D1%8C}{Берия, Лаврентий Павлович}

\pagebreak
\subsection{Экономическая политика в середине–второй половине 50-х годов: цели, методы, противоречия и результаты}

\pagebreak
\subsection{«Новый экономический курс» первой половины 60-х годов: идеология, направления, итоги}

\pagebreak
\subsection{Причины экономических трудностей начала 60-х, альтернативы нового этапа экономического реформирования}

\pagebreak
\subsection{Развитие советской науки и техники в 50-60-е годы}

Научно-технический прогресс оказал значительноевлияние на развитие советской науки. Особое внимание в области научных исследований в этот период уделялось теоретической физике. Достижения советских ученых-физиков получили широкое признание во всём мире. Крупнейшим советским физикам в эти годы были присуждены Нобелевские премии: П.А. Черенкову, И.Е. Тамму, И.М. Франку – за открытие и объяснение эффекта люминесцентного излучения Вавилова – Черенкова (1958год); Л.Д. Ландау за разработку теории жидкого геля (1962 год); Н.Г. Басову и А.М. Прохорову – за исследования в области квантовой электроники (создание лазера и мазера) (1964 год). Следует особо отметить, что с 1957 года были восстановлены Ленинские премии, присуждавшиеся ученым за выдающиеся достижения в области науки и техники.

\hfill

В 1957 году в СССР впервые был запущен самый мощный ускоритель элементарных частиц синхрофазотрон. Это практическое открытие позволило развивать новое направление науки – физику высоких и сверхвысоких энергий, что, в свою очередь, привело к созданию и развитию совершенно новых отраслей народного хозяйства – атомной промышленности и энергетики. В СССР в 1954 году впервые в мире была построена и запущена атомная электростанция в подмосковном научном городе Обнинске. Вслед за этой атомной электростанцией началось строительство более крупных: Воронежской, Белоярской и Сибирской АЭС. 1957 год ознаменовался спуском на воду первого в мире атомохода – ледохода «Ленин».

\hfill

Значительными открытиями были отмечены работы советских ученых в области автоматики и телемеханики, квантовой электроники, вычислительной техники и кибернетики, радиоэлектроники, физики полупроводников.

\hfill

На рубеже 1950 – 1960-х годов методы математического моделирования, разработанные В.С. Кулебякиным, Н.М. Крыловым, Н.Н. Боголюбовым, стали проникать в биологию, языкознание и даже историческую науку.

\hfill

Нобелевской премией в 1956 году были отмечены труды академика Н.Н. Семенова в области теории цепных реакций. Достижения в области теоретической химии позволили создать новые сверхпрочные материалы – полимеры.

\hfill

Наряду с физикой, химией и математикой развивалась также биология. Однако личная поддержка Н.С. Хрущевым академика Т.Д. Лысенко сдерживала развитие в СССР молекулярной биологии, генетики и генной инженерии.

\hfill

Эпоха «оттепели» вошла в историю мирового технического прогрессакак эра покорения космоса. Благодаря крупнейшим достижениям в области ракетно-космической техники СССР стал на многие годы признанным лидером в исследовании околоземного пространства. В СССР 4 сентября 1957 года на основе созданных советскими учеными конструкторами под руководством С.П. Королёва баллистических многоступенчатых ракет был произведён запуск первого в мире искусственного спутника Земли. Прошло всего несколько лет, и 12 апреля 1961 года впервые в истории человечества на космическом корабле «Восток» летчик-космонавт Юрий Алексеевич Гагарин совершил пилотируемый полёт по околоземной орбите, который продолжался 108 минут (1 час 48 минут). Вслед за полётом Ю.А. Гагарина за период с 1961 по 1963 год по программе «Восток» на кораблях этой серии было совершено еще шесть полётов. В 1965 году космонавт А.А.Леонов впервые осуществил выход в открытый космос.

\hfill

Помимо космической техники в СССР значительные успехи были достигнуты в области авиастроения. В 1956 году в небо поднялся первый турбореактивный пассажирский лайнер Ту-104, созданный в ОКБ под руководством А.Н. Туполева. В 1957 году коллективом инженеров и конструкторов во главе с С.В. Ильюшиным был создан четырёхмоторный турбовинтовой пассажирский самолёт Ил-18, ставший одним из наиболее надёжных самолётов отечественной гражданской авиации.

\hfill

Развитие космической и авиационной техники в СССР стало основой для расширения исследований астрономов и астрофизиков. С помощью искусственных спутников Земли советским ученым удалось исследовать внешний радиационный пояс и магнитное поле нашей планеты, осуществить фотографирование обратной стороны Луны, открыть новые астрофизические объекты.

\hfill

В период с середины 1950-х до начала 1960-х годов было опубликовано большое количество документальных сборников и мемуарной литературы, раскрывающих проблемы истории советского общества, ранее совершенно закрытые для исследователей. Наметились определённые позитивные изменения как в фундаментальных, так и в прикладных отраслях общественных наук. Отрадным явлением стало создание новых научных журналов: «Мировая экономика и международные отношения, «Вопросы истории КПСС», «История СССР», «Новая и новейшая история», «Вопросы языкознания».

\hfill

На страницах научной периодики в этот период разворачиваются дискуссии по актуальным проблемам теории и методологии советского обществоведения, идёт поиск новых подходов к решению базисных научных задач (например, обсуждение вопросов периодизации отечественной истории). В это время из небытия возвращаются имена незаконно репрессированных в годы сталинщины деятелей советского государства, партийных руководителей и военачальников Однако следует отметить, что внутренние противоречия, присущие периоду «оттепели» в сфере идеологии, не позволили советским обществоведам преодолеть концептуальный консерватизм и критически переосмыслить опыт социалистического строительства в СССР, выявить объективные причины зарождения и развития авторитаризма в нашей стране, поскольку некоторые попытки анализа всей системы социализма встречали решительный отпор.

\end{document}