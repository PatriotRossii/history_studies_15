\documentclass{article}
\usepackage[utf8]{inputenc}

\usepackage[T2A]{fontenc}
\usepackage[utf8]{inputenc}
\usepackage[russian]{babel}

\usepackage{multienum}
\usepackage{geometry}
\usepackage{hyperref}

\geometry{
    left=1cm,right=1cm,
    top=2cm,bottom=2cm
}

\title{История}
\author{Лисид Лаконский}
\date{May 2023}

\newtheorem{definition}{Определение}

\begin{document}
\raggedright

\maketitle
\tableofcontents
\pagebreak

\section{Практическое занятие по истории №16, «Социально-экономическое развитие СССР в 1953–1964 годы»}

\subsection{Экономика страны в начале 50-х годов}

\pagebreak
\subsection{В поисках новой стратегии развития: «генеральная линия» 1954 года и проекты Берии, Маленкова и Хрущева}

\pagebreak
\subsection{Экономическая политика в середине–второй половине 50-х годов: цели, методы, противоречия и результаты}

\pagebreak
\subsection{«Новый экономический курс» первой половины 60-х годов: идеология, направления, итоги}

\pagebreak
\subsection{Причины экономических трудностей начала 60-х, альтернативы нового этапа экономического реформирования}

\pagebreak
\subsection{Развитие советской науки и техники в 50-60-е годы}

\end{document}